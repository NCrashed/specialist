\pagenumbering{gobble}
\section*{Реферат}
%\pageref{LastPage}
Данная расчетно-пояснительная записка содержит 154 (без приложения) страниц, \total{figure} иллюстраций (без приложения), \total{table} таблиц, 1 приложение, 24 использованных источников.

Ключевые слова: полнотекстовый поиск, сбор информации, формальный язык запросов, генетическое программирование, эволюционные алгоритмы, символьная регрессия, прогноз временных рядов.

Данный дипломный проект посвящен разработке автоматизированной информационной системы для анализа новостных потоков, а также прогнозирования их активности. Целью разработки является сбор информации из открытых источников, накопление данной информации в СУБД, проведение анализа накопленных новостных данных на основе проблемно ориентированного языка программирования и предсказание количества новостей по определенной тематике с помощью эволюционных методов прогнозирования.

Система проектируется в виде веб-сервиса, предоставляющих пользователю набор экранных форм для взаимодействия с данным программных изделием, а также машинный интерфейс для взаимодействия стороннего программного обеспечения с данной АИС. Структуру системы составляют сервер приложения, сервер СУБД, сервер индексации и терминалы пользователей. При разработке программного продукта используется язык программирования Haskell.

В результате разработки была спроектирована автоматизированная информационная система, отвечающая требованиям технического задания и имеющая возможности, необходимые как для сбора и хранения новостной информации в базах данных, так и для всестороннего ее анализа.

Область применения АИС -- лаборатории анализа данных, составной компонент в распределённых системах анализа данных. Проект является некоммерческим с малой стоимостью выполнения.
