\section*{Определения, обозначения и сокращения}

\begin{itemize}
\item АПК -- аппаратно-программный комплекс.
\item БД -- база данных.
\item ЕСКД -- единая система конструкторской документации.
\item Управляющий класс (control) -- класс модели анализа, представляющий координацию, последовательность и управление другими объектами, часто используется для инкапсуляции управления для варианта использования.
\item Граничный класс (boundary) -- класс модели анализа, используемый для моделирования взаимодействия между системой и ее актантами, то есть пользователями и внешними системами.
\item Класс сущности (entity) -- класс модели анализа, используемый для моделирования долгоживущей, часто персистентной информации.
\item ОЗУ -- оперативное запоминающее устройство.
\item ОС -- операционная система.
\item ПК -- персональный компьютер.
\item ПО -- программное обеспечение.
\item ПЭВМ -- персональная электронно-вычислительная машина.
\item СанПиН -- санитарно-эпидемиологические правила и нормативы.
\item СИБИД -- система стандартов по информации, библиотечному и издательскому делу.
\item СНиП -- строительные нормы и правила. 
\item ССБТ -- система стандартов по информации, библиотечному и издательскому делу.
\item СУБД -- система управления базами данных.
\item ЭВМ -- электронно-вычислительная машина.
\item ЭМП -- электромагнитное поле.
\item API -- application programming interface -- набор готовых классов, процедур, функций, структур и констант, предоставляемых приложением (библиотекой, сервисом) для использования во внешних программных продуктах.
\item CPU -- central processing unit -- центральный процессор.
\item HDD -- hard disk drive -- жесткий диск.
\item SQL -- structured query language -- универсальный компьютерный язык, применяемый для создания, модификации и управления данными в реляционных базах данных. SQL основывается на исчислении кортежей. SQL является, прежде всего, информационно-логическим языком, предназначенным для описания, изменения и извлечения данных, хранимых в реляционных базах данных. 
\item UML -- unified modeling language, унифицированный язык моделирования -- язык графического описания для объектного моделирования в области разработки программного обеспечения. UML является языком широкого профиля, это открытый стандарт, использующий графические обозначения для создания абстрактной модели системы, называемой UML-моделью.
\end{itemize}