\subsection{Внутренне проектирование}
\subsubsection{Разработка структуры системы}

Разрабатываемая система является самостоятельным продуктом, но следует помнить о полезности модульного подхода для возможности расширения и модернизации. Функциональная архитектура системы отражены в графической части дипломного проекта на листе «Структурная схема».

\paragraph{Анализ информационных потоков} \hfill

Анализ информационных потоков выявил следующие источники и потребителей информации:
\begin{itemize}
\item Пользователь -- является одновременно и потребителем и источником информации. Пользователь создает формализованные запросы и определяет настройки системы, а потребляет данные, обработанные АИС;
\item Новостные сайты -- источники информации, с которых производится сбор сообщений;
\end{itemize}

\paragraph{Определение состава компонентов системы} \hfill

Согласно требованиям ТЗ касательно функциональности разрабатываемого комплекса, можно выделить следующие составляющие компоненты.

\subparagraph{Модуль рубрикатора новостей и сохранённых запросов.} \hfill

Должен предоставлять пользователю интерфейс для просмотра и редактирования рубрик (категорий) новостей, а также сохранение, редактирование и удаление сохранённых запросов, которые привязываются к рубрикам. Рубрики должны образовывать древовидную структуру, листами которой являются либо рубрики, либо сохранённые запросы. Для каждой рубрики должно отображаться количество новостей, относящихся к данной рубрике. Отсутствие рубрики у сохранённого запроса или у новости должно обрабатываться и отображаться пользователю как специальная рубрика «Нерубрицированные».

Входные данные:
\begin{itemize}
\item Действия над рубриками и запросами: добавление, удаление, обновление;
\item Данные рубрик: название и положение в дереве из рубрик;
\item Данные сохранённых запросов: тело запроса на \\ проблемно-ориентированном языке, дополнительные фильтры на дату публикации, источник и теги документаю
\end{itemize}

Выходные данные:
\begin{itemize}
\item Рубрикатор с рубриками и количеством документов в них;
\item Сохранённые запросы, прикреплённые к рубрикам;
\end{itemize}

\subparagraph{Модуль поиска новостей и формализованных запросов.}  \hfill

Должен предоставлять пользователю интерфейс для ввода поискового запроса на
естественном языке или на формализованном языке запросов. Поиск должен
иметь возможность:
\begin{itemize}
\item настройки сортировки по времени новости и её релевантности запросу;
\item указания интервала времени для поиска;
\item дополнительной фильтрации по реквизитам документа;
\item поиска с условием наличия/отсутствия конкретных меток новостей.
\end{itemize}

Формализованные поисковые запросы должны иметь возможность:
\begin{itemize}
\item Поиска с учётом морфологии Русского и Английского языков;
\item Поиска с учётом максимального расстояния между ключевыми словами;
\item Поиска фиксированной фразы;
\item Поиска по конкретному реквизиту документа;
\item Поиска по ключевым словам, расположенных в конце и начале реквизита;
\item Поиска по ключевым словам с указанием приоритета для каждого из них;
\item Поиска в пределах одного предложения или параграфа;
\item Поиска с составными запросами, части которого объединены логическим оператором «ИЛИ».
\end{itemize}

Входные данные:
\begin{itemize}
\item Запрос на проблемно-ориентированном языке.
\end{itemize}

Выходные данные:
\begin{itemize}
\item Список документов, удовлетворяющих запросу. Текст документа должен содержать разметку, указывающую на части текста, которые соответствуют запросу.
\item Количество документов, всего удовлетворяющих запросу.
\item Время выполнения запроса.
\end{itemize}

\subparagraph{Модуль просмотра новости.} \hfill

Должен предоставлять пользователю полную информацию о документе (новости) и его реквизитах. Также модуль должен предоставлять возможность редактировать реквизиты документа (за исключением идентификатора и источника) и возможность удаления новости из АИС.

Реквизиты документа:
\begin{itemize}
\item Заголовок новости -- краткий заголовок новости;
\item Основная часть новости -- основной массив текста с форматированием;
\item Время публикации новости -- время публикации новости, указанное в
источнике;
\item Рубрика новости -- категория новости, к которой она относится;
\item Источник новости -- адрес сайта новости;
\item Идентификатор новости -- уникальный идентификатор новости, под
которым она хранится в БД.
\item Метки новости -- список строк-меток, которые были присвоены новости;
\end{itemize}

Входные данные:
\begin{itemize}
\item Запрос на проблемно-ориентированном языке для подсветки основной части документа. Может отсутствовать, тогда отображается весь текст документа без подсветки.
\item Идентификатор документа для отображения.
\item Обновленные значения реквизитов документа при их редактировании.
\end{itemize}

Выходные данные:
\begin{itemize}
\item Реквизиты документа, перечисленные выше.
\item Обновлённые данные в базе данных и индексе при редактировании полей документа.
\end{itemize}

\subparagraph{Модуль управления индексом.} \hfill

Должен предоставлять пользователю интерфейс для совершения следующих
операций над поисковым индексом:
\begin{itemize}
\item Пересоздание поискового индекса;
\item Синхронизация поискового индекса -- проверка и добавления отсутствующих документов в индексе;
\item Удаление дублей документов;
\item Перемещение документов между рубриками; 
\item Экспорт документов из рубрики или по результатам запроса;
\item Импорт документов из архива;
\item Удаление документов по сохранённому запросу;
\end{itemize}

Модуль должен предоставлять возможность просмотра запущенных операций над поисковым индексом и преждевременного завершения этих задач. При включении АИС должна продолжить выполнение операций, которые были в процессе исполнения во время завершения работы АИС.

Входные данные:
\begin{itemize}
\item Команда старта/остановки/перезапуска операции;
\end{itemize}

Выходные данные:
\begin{itemize}
\item Список операций с временем их старта и конца, результатом и прогрессом выполнения;
\end{itemize}

\subparagraph{Модуль интеграционного интерфейса.} \hfill

Должен предоставлять пользователю интерфейс для взаимодействия с АИС других программ по протоколу HTTP, используя формат JSON. В интеграционный интерфейс должны входить следующие операции:
\begin{itemize}
\item Получение версии интерфейса;
\item Выполнение формализованных запросов;
\item Добавление, обновление и удаление документов (новостей);
\item Переиндексирование документа;
\item Добавление меток документу;
\item Создание рубрики и получения детальной информации о рубрике;
\item Получение дерева рубрикатора;
\item Перемещение документов между рубриками;
\item Создание, удаление и выполнение сохранённых запросов;
\item Получение детальной информации о сохранённом запросе;
\item Получение списка сохранённых запросов;
\item Получение и задание настроек системы;
\item Получение информации о текущей операции над поисковым индексом.
\end{itemize}

Входные данные:
\begin{itemize}
\item Операция интеграционного интерфейса с правильно сформулированными аргументами в формате JSON;
\end{itemize}

Выходные данные:
\begin{itemize}
\item Результат выполнения операции в формате JSON;
\item Сообщения об ошибке в формате JSON или через коды состояний HTML.
\end{itemize}

\subparagraph{Модуль настроек системы.} \hfill

Должен предоставлять интерфейс для изменения следующих настроек системы:
\begin{itemize}
\item Отключение/Включение автоматического добавления документов;
\item Показ скрытых рубрик;
\item Показ отладочной информации об используемом формализованном запросе при поиске;
\item Показ всех полей в форме поиска документов;
\end{itemize}

Входные данные:
\begin{itemize}
\item Новые значения настроек;
\end{itemize}

Выходные данные:
\begin{itemize}
\item Обновленные данные о настройках в базе данных;
\item Графически формы с отображением текущих значений настроек системы;
\end{itemize}

\subparagraph{Модуль добавления документов.} \hfill

Должен предоставлять пользователю интерфейс добавления документов в АИС тремя способами:
\begin{itemize}
\item Добавление через HTML форму ввода -- реквизиты документа вводятся в поля формы, проходят проверку на соответствие формату входных данных и отправляются на сервер, где новый документ добавляется в БД и поисковый индекс.
\item Добавление через директорию импорта -- автоматический метод добавления документов, который осуществляется через сканирование специальной директории на наличие файлов, содержащих документы в формате XML или JSON. При успешном добавлении документ перемещается в специальную директорию для добавленных документов. При неудачном добавлении документа в АИС, документ перемещается в специальную директорию для некорректных документов.
\item Добавление архива документов -- отправка архива с документами на сервер. Архив формируется вручную пользователем или с помощью функции экспорта документов. После отправки на сервер архив распаковывается, и все документы из него загружаются в базу данных и индексатор. Данный метод позволяет загружать документы как в формате XML, так и в формате JSON.
\end{itemize}

Входные данные:
\begin{itemize}
\item Реквизиты документов: единичного документа через форму, нескольких документов через папку импорта, нескольких документов через импорт архива.
\end{itemize}

Выходные данные:
\begin{itemize}
\item Добавленные данные о документах в базе данных и индексе;
\item Сообщения об ошибках загрузки документов;
\end{itemize}

\subparagraph{Модуль прогнозирования.} \hfill

Должен предоставлять возможность анализировать графики количества документов, соответствующих сохранённым запросам, по дням и проводить кратковременный прогноз этого количества документов для будущих дат. Данный модуль должен предоставлять следующую информацию пользователю:
\begin{itemize}
\item График количества документов, соответствующих сохранённому запросу, используемого для прогнозирования, по дням до текущей даты;
\item График прогнозируемого количества документов, соответствующих сохранённому запросу, используемого для прогнозирования, по дням на всём рассматриваемом интервале времени;
\item Аналитическую формулу прогноза, по которой значение количества документов, соответствующих сохранённому запросу, можно вычислить для любого дня рассматриваемого интервала времени;
\item Количественную оценку полученного прогноза;
\end{itemize}

Входные данные:
\begin{itemize}
\item Параметры прогноза: количество и размер популяции для эволюционного алгоритма, ссылка на сохраненный формализованный запрос для прогнозирования.
\item Команды на сброс данных прогноза или данных формализованного запроса;
\end{itemize}

Выходные данные:
\begin{itemize}
\item Графики прогноза и сохраненного запроса;
\item Аналитическая формула прогноза;
\item Оценка качества прогнозирования;
\end{itemize}

\subparagraph{Модуль индексирования.} \hfill

Должен предоставлять возможность другим модулям выполнять полнотекстовые запросы и проводить индексирование документов для осуществления полнотекстового поиска.

Входные данные:
\begin{itemize}
\item Реквизиты документа для индексирования;
\item Текст формализованного запроса для выполнения.
\end{itemize}

Выходные данные:
\begin{itemize}
\item Структура индекса, сохраняемая на жесткий диск ЭВМ;
\item Список документов, удовлетворяющих запросу;
\item Метаинформация по запросу: время выполнения и общее количество документов, удовлетворяющих запросу;
\end{itemize}