\section{Технологическая часть}

\subsection{Общее описание программного комплекса}
\subsubsection{Функциональное назначение}

Разработанная АИС предназначена для автоматизации перечисленных процессов:
\begin{itemize}
\item Ввод документа в систему;
\item Построение ускоряющего поиск индекса;
\item Выполнение полнотекстового поиска на основе формализованного языка запросов;
\item Просмотр документов и их редактирование;
\item Выполнение операций над индексом;
\item Рубрикация документов и сохранённых запросов;
\item Прогнозирование новостных потоков.
\end{itemize}

Разработанная АИС должна работать в режиме веб-приложения, в соответствии с выбранной на этапе проектирования архитектурой. Процесс развертывания системы не должен быть чрезмерно сложен или требовать нереалистичных объемов времени.

\subsubsection{Средства технического обеспечения}

Для корректного функционирования программного комплекса необходимы следующие технические средства, имеющие характеристики не меньше указанных.

\paragraph*{Сервер} \hfill

Персональная ЭВМ архитектуры AMD x86\_64:
\begin{itemize}
\item процессор Intel i7-860 (4 ядра, 2.8 ГГц);
\item жесткий диск не менее 1 Тб;
\item оперативная память 4 Гб;
\item сетевой адаптер для подключения к ЛВС.
\end{itemize}

\paragraph*{Сервер индексации} \hfill

Персональная ЭВМ архитектуры AMD x86\_64:
\begin{itemize}
\item процессор Intel i7-860 (4 ядра, 2.8 ГГц);
\item жесткий диск не менее 500 Гб;
\item оперативная память 8 Гб;
\item сетевой адаптер для подключения к ЛВС.
\end{itemize}

\paragraph*{Клиентская машина} \hfill

Персональная ЭВМ архитектуры AMD x86\_64 или IBM x86:
\begin{itemize}
\item процессор Intel Pentium Dual-Core;
\item жесткий диск не менее 100 Гб;
\item оперативная память 2 Гб;
\item сетевой адаптер для подключения к ЛВС.
\item Клавиатура, мышь, экран;
\end{itemize}


\subsubsection{Программное обеспечение, необходимое для функционирования}

Для функционирования АИС требуется следующее ПО.

  \paragraph*{Серверное ПО} \hfill

  \begin{itemize}
  \item Операционная система CentOS 7;
  \item СУБД PostgreSQL версии 9.4;
  \item Библиотека gmp версии 3 для работы с целочисленными числами произвольной точности;
  \item Библиотека pcre версии 3 для работы с регулярными выражениями;
  \item Клиентская библиотека от MySQL для протокола связи между сервером приложения и сервером индексации;
  \item Библиотека expat для работы с XML файлами;
  \end{itemize}

  \paragraph*{Клиентское ПО} \hfill

  \begin{itemize}
  \item Любая ОС;
  \item Браузер Firefox версии 46 или Chrome версии 51;
  \end{itemize}

\subsection{Структура программы с описанием функций составных частей}

Разрабатываемая АИС запаковывается в пакеты формата \textbf{deb}. Система поставляется в трех пакетах:
\begin{itemize}
\item volchv-ips-server -- сервер приложений АИС.
\item volchv-ips-indexer -- сервер индексации АИС.
\item volchv-predictor -- сервис прогнозирования АИС.
\end{itemize}

\subsubsection{Архив сервера приложений}  \hfill

Архив сервера приложений состоит из:
\begin{itemize}
\item \textbf{etc/init.d/volchv-ips-server} -- конфигурационный файл сервиса ОС для сервера приложений. 
\item \textbf{etc/volchv/ips-server.yml} -- конфигурационный файл сервера приложений.
\item \textbf{usr/bin/ips} -- исполняемый файл сервера приложений, который устанавливается как сервис ОС. 
\item \textbf{var/lib/volchv/ips-server/static} -- статичные файлы сервера приложений, содержащие HTML, CSS и JavaScript файлы клиентской части АИС.
\end{itemize}

\subsubsection{Архив сервера индексации} \hfill

Архив сервера индексации состоит из:
\begin{itemize}
\item \textbf{etc/init.d/volchv-ips-indexer} -- конфигурационный файл сервиса ОС для сервера индексации. 
\item \textbf{etc/volchv/ips-indexer} -- конфигурационные файлы и словари русского и английского языков.
\item \textbf{usr/bin} -- исполняемые файлы сервера индексации, необходимые для разбора слов, поиска по словарям, отладки и обработки формализованныйх запросов. 
\item \textbf{usr/share} -- файлы документации для внутренних инструментов отладки сервера индексации.
\end{itemize}

\subsubsection{Архив сервиса прогнозирования} \hfill

Архив сервиса прогнозирования состоит из:
\begin{itemize}
\item \textbf{etc/init.d/volchv-predictor} -- конфигурационный файл сервиса ОС для сервиса прогнозирования. 
\item \textbf{etc/volchv/predictor.yml} -- конфигурационный файл сервиса прогнозирования.
\item \textbf{usr/bin/predictor} -- исполняемый файл сервера приложений, который устанавливается как сервис ОС. 
\item \textbf{var/lib/volchv/predictor/static} -- статичные файлы сервиса прогнозирования, содержащие HTML, CSS и JavaScript файлы дополнительной клиентской части АИС для поддержки функций прогнозирования.
\end{itemize}

\subsection{Установка и запуск приложения}

Для установки АИС следует провести следующие действия:
\begin{enumerate}
\item Установить все необходимое ПО на серверах для сервера приложений и сервера индексирования.
\item Выполнить подключение репозиториев с пакетами CentOS согласно руководству системного администратора CentOS.
\item С помощью программы \textbf{dpkg} и \textbf{apt-get} произвести установку пакетов АИС. Установка производится согласно руководству системного администратора CentOS.
\item В конфигурационных файлах, перечисленных выше, сервера приложений и сервера индексации указать взаимные адреса серверов.
\item Проверить работоспособность системы и приступить к её эксплуатации.
\end{enumerate}
